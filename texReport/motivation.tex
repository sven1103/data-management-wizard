
\section{Introduction and Motivation}
The increasing amount of data and the participation of several institutions in a project makes it important to document well how to handle essential aspects like the storage, analysis and integration of data during the project. This can be realized with a Data Management Plan \cite{lecture}. Moreover, this plan ensures before the data collection starts that data is in the correct format, well-organized and well annotated \cite{planWhy}. The documentation of the different steps throughout the data's life cycle helps data users to understand and use the data in the future. 

Furthermore, the data management plan also makes the data available to other researchers upon project completion, which can impinge positive on the whole work, concerning discovery and relevance \cite{planWhy}.


There exists no standardized guidance how to create a data management plan however the DAMA Data Management Body of Knowledge \cite{DAMAInternational:2009:DGD:1593444} provides a good orientation of essential aspects which should be part of the plan \cite{lecture}. 


During the scope of the project, it was our task to develop a Data Management Planning Tool. The tool should be able to creates automatically a DMP based on an experimental design given as a .tsv file. This file was generated by \textit{QWizard} \cite{qwizard}, a portlet to input experimental data. The tool also offers users the possibility to add project information which are not included in the .tsv file.


The following chapters give an overview about the tool \textit{CMPcreator} which was implemented during the scope of this project. The last chapters compare our tool with the  already existing tool \textit{DMPTool} developed by the University of California\footnote{\url{https://dmptool.org/}} followed by an outlook how our tool can be extended.

