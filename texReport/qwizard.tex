\section{Experimental Design Knowledge}
One of the project's task was the implementation of results from an web-based experimental design tool called \texttt{QWizard} \cite{qwizard}. It is part of QBiC's web-based science gateway \texttt{QPortal} which handles scientific experimental projects and data. Customers are using the \texttt{QWizard} for setting up their experimental design and therefore provide information which already can be used for the data management plan. To name some of the information that will be available from the \texttt{QWizard} are detailed description of the biological entity, such as treatment or phenotype, as well as the species. Also the sample extraction is described, listing the type of tissue used as well as defined conditions used. Moreover, the \texttt{QWizard} will provide information of the sample size and technologies used for retrieving the data.

This information also belongs into the data management plan as it is part of the experimental design and therefore essential for the project planning and execution. We can also use the information for supporting the user during the process of generating a data management plan by suggesting storage amount and backup solutions for example.

In order to use the information given by the \texttt{QWizard}, we will provide an upload section, where the user can upload the file with all the content generated from \texttt{QWizard} and a background parser will retrieve the information as well as automatically implementing it in the report template.