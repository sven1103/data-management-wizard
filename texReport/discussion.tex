\section{Discussion}
Proper data management will become more and more important in the future, as new developments of technologies are going to produce an expanding amount of data. This is in particular true for scientific applications. The recent developments in next-generation sequencing and mass spectrometer analysis have shown, that there is a flood of data that is created during a project's lifetime.
Initiatives have been started over the past years, to formulate standards in data management. The DAMA International Foundation is an example for such an initiative, that took steps towards an important direction. It has defined the Data Management Book of Knowledge (DMBOK), which is a guideline for proper data management and categorizes the task in eleven different fields. 

During the lecture, we got to know the Data Management Plan Tool (DMPtool) from the University of California, which provides a guided web-interface for generating a data management plan. We experienced the tool as a step in the right direction, but remark that it is not standardized at all, concerning the input the user can do in order to create such a plan. There are a lot of text-fields, which allow the input of free text, but this raises issues at the same time. The user can input what he wants, and there is no standard for that, nor do inexperienced user have a proper idea what to fill in. Though the tool provides templates, which are plans created from other users, the quality of the input finally relies on the user.

With our tool DMPcreator, we want to present a data management plan tool that follows the DMBOK guideline and controls some of the standards mentioned there. Additionally, we provide an integration for projects created with QWizard, a web-based experimental design tool from the Quantitative Biology Center (University of Tuebingen). The user can upload the project description as tabbed-separated file (tsv), and this content will be parsed automatically and information such as storage requirement derived from the technology type will be integrated in the web-interface as well as in the report automatically. The user can still change the input, if necessary.
The taxonomy ID in the tsv-file will be queried automatically from the NCBI web-server and translated in the organism species.
We used Vaadin as a Java framework and provide a user-friendly interface with intuitive and customizable components such as flexible drop-down menus with predefined content and dynamic tables that will display the user input. Because of the short project time, we only implemented topics from the DMBOK such as 'Storage and Backup', 'Documentation Management', 'Dissemination', 'Roles and Responsibilities' and 'General Information' for contact data. Of course, in order to provide a detailed data management plan a lot more chapters and options have to be added. We designed the software as generic as possible, developers can add more user slides and extend the class for parsing the content into the report as pdf.

We have shown, that a more user-friendly and more standardized data management plan tool than DMPtool is possible, and has to be the further development. The DMBOK provides the ideal template for such a tool and frameworks offer the components and functionality to build a detailed and guided interface to generate a content-rich and standardized data management plan.