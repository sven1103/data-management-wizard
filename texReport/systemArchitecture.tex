\section{System Architecture}
\subsection{Java Framework}
For the setup of an own data management planning tool we decided to use the \texttt{Vaadin Framework}\footnote{\url{https://vaadin.com/home}}. \texttt{Vaadin} is a single-page web framework for Java developers that provides powerful functions for creating rich Internet applications (RIAs) without the knowledge of classical web-languages as HTML5 or CSS3. It also provides a huge library of functional components that can be included in the project and already satisfy out needs for usability and navigation.
The \texttt{Vaadin Framework} takes care of browser incompatibilities and automatically designs ajax communication protocols, which we evaluated as advantage in terms of time efficiency during the development. It is also open-source but still provides a support service. 

As we are all quite familiar with Java, the availability of a huge component library, good documentation and support as well as the capacity to build a rich Internet application convinced us to give it a chance and use it for our own creation of a data management planning tool.

\subsection{Java Application Server}
In order to run the project, local or on a web-server, we needed to set-up a Java application server. There are numerous servers available, most of them are free and open-source. For our needs, we thought it does not make a big difference which server we select as we will not use the full capacities anyway. 
Apache's Tomcat was already known to us and was used before in other projects. To get to know a new application server which also comes with a nice documentation and web interface is JBoss' Wildfly\footnote{\url{http://wildfly.org/}}. 
In our case the installation and set-up of Wildfly was easy under Linux OS. So we configured local instances on every development environment and used the server's default settings.

\subsection{Vaadin Theme}
The \texttt{Vaadin Framework} also comes with additional themes, that apply a different layout and style to the graphical interface elements. We decided to choose the theme 'Valo', as it comes with a complete set of designed components, is responsive and ensures a pleasant user experience.